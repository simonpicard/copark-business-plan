\documentclass[12pt,a4paper,oneside]{book}
\usepackage{setspace}
\onehalfspacing
\usepackage[hmargin={2.5cm,2.5cm},vmargin={2.5cm,2.5cm}]{geometry}


\usepackage[english]{babel}
\usepackage[utf8]{inputenc}
\usepackage[T1]{fontenc}

\usepackage{graphicx}

\usepackage[final]{pdfpages} % include pdf files

\usepackage{amsthm}
\usepackage{amsmath}
\usepackage{amsfonts}
\usepackage{tabularx}
\usepackage[nottoc, notlot, notlof, numbib, numindex]{tocbibind}
\usepackage{charter}


\usepackage{url}
%\usepackage{geometry}
\usepackage{enumitem}
\usepackage{hyperref}
\usepackage{comment}
\usepackage{listings} 
\usepackage{caption} 
\usepackage{stmaryrd}
\usepackage{fancybox}
\usepackage{makeidx}
\usepackage{eurosym}


\begin{document}
%\maketitle


\begin{titlepage}
\noindent \begin{minipage}{0.83\textwidth}
\noindent \textbf{UNIVERSIDAD CARLOS III DE MADRID}\hfill{}\\
\textbf{Graduate School of Business}\hfill{}\\
\textbf{Master in Management}\hfill{}
\end{minipage}
\begin{minipage}{0.17\textwidth}
\includegraphics[keepaspectratio=true,width=\textwidth]{images/logo_UC3M_universidad_Carlos_III_Madrid.jpg}
\end{minipage}
\begin{center}
\vfill{}\vfill{}\vfill{}
%\begin{center}
{\Huge Business Plan : Park your car}
%\end{center}
{\Huge \par}
\begin{center}{\LARGE Simon \textsc{Picard}}\end{center}{\Huge \par}
%\vfill{}\vfill{}\vfill{}\vfill{}\vfill{}
\vfill{}
\includegraphics[keepaspectratio=true,width=\textwidth-2cm]{images/Seal_of_the_University_of_Carlos_III.jpg}
\vfill{}
\begin{flushleft}{\large \textbf{Supervisor  :}}\\
{\large Prof. Kurt \textsc{Achiel Desender}}
\end{flushleft}{\large\par}
\vfill{}\vfill{}\enlargethispage{2cm}
\textbf{Academic year 2016~-~2017}
\end{center}
\end{titlepage}



\newpage
%\newgeometry{hmargin={3.5cm,1.5cm},vmargin={2.5cm,2.5cm}}
\thispagestyle{empty} 
\null


\tableofcontents

\chapter{Executive summary}

\chapter{Industry environment}

\section{Macro environmental analysis}

\subsection{Economic}
According to the GDP per capita (PPP), the European Union is the second largest economy in the world.Its total GDP is \euro 16.5 trillions in 2016 , which represent 22.8\% of the global GDP\cite{imfgdp}. The 2008 financial crisis appears to be in the process of recovery, as the economy of 19 of its countries advanced by 0.6 percent over the first three months of the year, as compared to the previous quarter\cite{eurorecov}.\\

Belgium's economy is mainly composed from the service sector, accounting for 74.9\% of its GDP. Belgium also profit from an heavy industrial sector, which is concentrated mainly in northern Flanders, around Brussels and in Liège and Charleroi. The industry sector represent 21.1\% of Belgium's GDP. Finally, the agriculture sector represent less than 1\% of its GDP.\\
With a GDP of \$508.6 billion (PPP, 2016), Belgium ranks itself at the 38th position of the richest country in the world. In 2016, the GDP growth was 1.4\% and the unemployment rate was 8.4\%.\\
With approximately two-third of Belgium's GDP relying on exportation, the country rely heavily on world trade. This high proporation comes from the higly its skilled, multi-cultural and central population\cite{ciafb}.\\

The economy of Brussels is mainly oriented around the service industry, with 88\% of all jobs being in the service sector. Brussels alone contribute to a fifth of Belgium's GDP. Brussels holds 550 000 jobs and it represent 17.7\% of the country employment. There is 2000 foreign companies offices in the capital.\cite{bxinfo}.\\
Brussels is one of the richest city of the world with a GDP per capita of 67,811 (PPP) in 2016, which ranks it at the 9th position within the raking of the city of the OECD\cite{oecdstat}, Brussels is thus the economic capital of the country. Brussels GDP is boosted by a number of commuter from nearby regions. There is 230 000 employee coming from Flanders and another 130 000 coming from Wallonia working in Brussels. On the other hand, only 16\% of Brussels habitant work outside of the city\cite{euresCom}.\\
Although having apparently a big wealth, Brussels is not the holder of all of it. Indeed, it appears that the proportion of the unemployed resident of Brussels was 20.4\% in December 2013\cite{unemploybx}.

\subsection{Political and legal}

Belgium is a constitutional, popular monarchy and a federal parliamentary democracy. The country is divided in three regions, Walloon, Flanders and Brussels, which all have regional government.\\
The worldwide governance indicator gives us informations about the political situation of Belgium. The indicators say that the Belgium is a country of low corruption, high government effectiveness, high regulatory quality, high rule of law and high voice and Accountability. Based on those indicators, Belgium lacks political stability and absence of violence/terrosrism with a value in this indicator of 65 in 2015\cite{wbgi}. On the other hand, when compared with other western European countries such as Spain and France, Belgium is valued higher in this field.\\
Part this political stability score can be explained by the language and regional division of the country and subsequently political opinions.\cite{bailo2016political}.\\

The Belgian employment laws are based on the consultation fo employee and workers. The lenght of the work is limited to 8 hours per day and 40 hours per week. There is a minimal wage which is valued at \euro 1 501,82 (2015)\cite{eurostatmw}.\\
A new potential law known as "Peteers' Law" is beeing studied, which would make the maximal number of hours that one can work based annualy instead of weeky. This would lead to a potential week of 45 hours.\cite{rtlp} This process show a trend of work deregulation.\\

As a member of the European Union, Belgium applies the "common customs tariff of the European Union" to goods imported from non-EU countries. Overall, Belgium is open to trade since the country itself rely heavily on it. The trade regulation are decreasing, as one can see from the rise of the European Union, CETA and TAFTA//TTIP. Although those regulation are not currently established, the trend is to facilitate the international trade, specially for Brussels, as the capital of Europe.\\

\subsection{Social}

Belgium as a total population of 11 250 000 (2016) habitant and it is growing at a rate of 0.82\% (2008). 66.3\% of the population is 15 and 64 years old. Its largest city is Brussels, with a total population of 1 175 173\cite{ciafb}.\\

An actual social concern for Belgium is the integration of second and third generation of immigrant. A part of this segment is not integrated to the country whether socioeconomicly of culturally. This phenomenon concern principally groups of young Belgian citizen of Moroccan origin who feel excluded from the society, this is particularly happening in Brussels and Antwerp\cite{sgikc}.\\

As Belgium has 27,3\% of its population younger than 24 years old and its median age is 43.1 years, Belgium has a younger population emerging\cite{ciafb}. It is then relevant to investigate the trends for the millennials and surrounding generation to understand the rises of trends in Belgium. The millenials are higly connected through social media and mobile data. The generation Z is even more. As the first members of the generation Z will turn 21 years old in 2017, their influence will impact the market.\\
This super-connection leads to promotion over social media, platformless shopping and digitalisation. Overall, what is needed is a fast process, with instant notification and picture based description of product\cite{stbe}.



\subsection{Environmental}

Belgium has high density of population which impacts its environment particularly in the major cities od the country. Overall, Belgium is oriented toward an environment friendly approach, being ranked 41 out of 180 in the environment protection index\cite{epi} and Belgium has one the most efficient recycling process, in Flanders, 75 \% of the residential waste produced are reused, recycled or composted\cite{wastemana}.

\subsection{Technological}



\chapter{Company and product description}

\chapter{Marketing plan}

\chapter{Operation and team}

\chapter{Critical risks}

\chapter{Financial plan}


\bibliographystyle{unsrt}
%\bibliographystyle{plain}
\bibliography{./reference/bibliography}


\end{document}
