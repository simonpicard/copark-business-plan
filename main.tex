\documentclass[12pt,a4paper,oneside]{book}
\usepackage{setspace}
\onehalfspacing
\usepackage[hmargin={2.5cm,2.5cm},vmargin={2.5cm,2.5cm}]{geometry}


\usepackage[english]{babel}
\usepackage[utf8]{inputenc}
\usepackage[T1]{fontenc}

\usepackage{graphicx}

\usepackage[final]{pdfpages} % include pdf files

\usepackage{amsthm}
\usepackage{amsmath}
\usepackage{amsfonts}
\usepackage{tabularx}
\usepackage[nottoc, notlot, notlof, numbib, numindex]{tocbibind}
\usepackage{charter}


\usepackage{url}
%\usepackage{geometry}
\usepackage{enumitem}
\usepackage{hyperref}
\usepackage{comment}
\usepackage{listings} 
\usepackage{caption} 
\usepackage{stmaryrd}
\usepackage{fancybox}
\usepackage{makeidx}
\usepackage{eurosym}
\usepackage{multirow}

\newcommand{\bp}{\textit{Park~Your~Car }}


\begin{document}
%\maketitle


\begin{titlepage}
\noindent \begin{minipage}{0.83\textwidth}
\noindent \textbf{UNIVERSIDAD CARLOS III DE MADRID}\hfill{}\\
\textbf{Graduate School of Business}\hfill{}\\
\textbf{Master in Management}\hfill{}
\end{minipage}
\begin{minipage}{0.17\textwidth}
\includegraphics[keepaspectratio=true,width=\textwidth]{images/logo_UC3M_universidad_Carlos_III_Madrid.jpg}
\end{minipage}
\begin{center}
\vfill{}\vfill{}\vfill{}
%\begin{center}
{\Huge Business Plan : Park your car}
%\end{center}
{\Huge \par}
\begin{center}{\LARGE Simon \textsc{Picard}}\end{center}{\Huge \par}
%\vfill{}\vfill{}\vfill{}\vfill{}\vfill{}
\vfill{}
\includegraphics[keepaspectratio=true,width=\textwidth-2cm]{images/Seal_of_the_University_of_Carlos_III.jpg}
\vfill{}
\begin{flushleft}{\large \textbf{Supervisor  :}}\\
{\large Prof. Kurt \textsc{Achiel Desender}}
\end{flushleft}{\large\par}
\vfill{}\vfill{}\enlargethispage{2cm}
\textbf{Academic year 2016~-~2017}
\end{center}
\end{titlepage}



\newpage
%\newgeometry{hmargin={3.5cm,1.5cm},vmargin={2.5cm,2.5cm}}
\thispagestyle{empty} 
\null


\tableofcontents

\chapter{Executive Summary}

\chapter{Industry Environment}

\section{Macro Environmental Analysis}

\subsection{Economic}
According to the GDP per capita (PPP), the European Union is the second largest economy in the world.Its total GDP is \euro 16.5 trillions in 2016 , which represent 22.8\% of the global GDP\cite{imfgdp}. The 2008 financial crisis appears to be in the process of recovery, as the economy of 19 of its countries advanced by 0.6 percent over the first three months of the year, as compared to the previous quarter\cite{eurorecov}.\\

Belgium's economy is mainly composed from the service sector, accounting for 74.9\% of its GDP. Belgium also profit from an heavy industrial sector, which is concentrated mainly in northern Flanders, around Brussels and in Liège and Charleroi. The industry sector represent 21.1\% of Belgium's GDP. Finally, the agriculture sector represent less than 1\% of its GDP.\\
With a GDP of \$508.6 billion (PPP, 2016), Belgium ranks itself at the 38th position of the richest country in the world. In 2016, the GDP growth was 1.4\% and the unemployment rate was 8.4\%.\\
With approximately two-third of Belgium's GDP relying on exportation, the country rely heavily on world trade. This high proporation comes from the higly its skilled, multi-cultural and central population\cite{ciafb}.\\

The economy of Brussels is mainly oriented around the service industry, with 88\% of all jobs being in the service sector. Brussels alone contribute to a fifth of Belgium's GDP. Brussels holds 550 000 jobs and it represent 17.7\% of the country employment. There is 2000 foreign companies offices in the capital.\cite{bxinfo}.\\
Brussels is one of the richest city of the world with a GDP per capita of 67,811 (PPP) in 2016, which ranks it at the 9th position within the raking of the city of the OECD\cite{oecdstat}, Brussels is thus the economic capital of the country. Brussels GDP is boosted by a number of commuter from nearby regions. There is 230 000 employee coming from Flanders and another 130 000 coming from Wallonia working in Brussels. On the other hand, only 16\% of Brussels habitant work outside of the city\cite{euresCom}.\\
Although having apparently a big wealth, Brussels is not the holder of all of it. Indeed, it appears that the proportion of the unemployed resident of Brussels was 20.4\% in December 2013\cite{unemploybx}.

\subsection{Political and legal}

Belgium is a constitutional, popular monarchy and a federal parliamentary democracy. The country is divided in three regions, Walloon, Flanders and Brussels, which all have regional government.\\
The worldwide governance indicator gives us informations about the political situation of Belgium. The indicators say that the Belgium is a country of low corruption, high government effectiveness, high regulatory quality, high rule of law and high voice and Accountability. Based on those indicators, Belgium lacks political stability and absence of violence/terrosrism with a value in this indicator of 65 in 2015\cite{wbgi}. On the other hand, when compared with other western European countries such as Spain and France, Belgium is valued higher in this field.\\
Part this political stability score can be explained by the language and regional division of the country and subsequently political opinions.\cite{bailo2016political}.\\

The Belgian employment laws are based on the consultation fo employee and workers. The lenght of the work is limited to 8 hours per day and 40 hours per week. There is a minimal wage which is valued at \euro 1 501,82 (2015)\cite{eurostatmw}.\\
A new potential law known as "Peteers' Law" is beeing studied, which would make the maximal number of hours that one can work based annualy instead of weeky. This would lead to a potential week of 45 hours.\cite{rtlp} This process show a trend of work deregulation.\\

As a member of the European Union, Belgium applies the "common customs tariff of the European Union" to goods imported from non-EU countries. Overall, Belgium is open to trade since the country itself rely heavily on it. The trade regulation are decreasing, as one can see from the rise of the European Union, CETA and TAFTA//TTIP. Although those regulation are not currently established, the trend is to facilitate the international trade, specially for Brussels, as the capital of Europe.\\

\subsection{Social}

Belgium as a total population of 11 250 000 (2016) habitant and it is growing at a rate of 0.82\% (2008). 66.3\% of the population is 15 and 64 years old. Its largest city is Brussels, with a total population of 1 175 173\cite{ciafb}.\\

An actual social concern for Belgium is the integration of second and third generation of immigrant. A part of this segment is not integrated to the country whether socioeconomicly of culturally. This phenomenon concern principally groups of young Belgian citizen of Moroccan origin who feel excluded from the society, this is particularly happening in Brussels and Antwerp\cite{sgikc}.\\

As Belgium has 27,3\% of its population younger than 24 years old and its median age is 43.1 years, Belgium has a younger population emerging\cite{ciafb}. It is then relevant to investigate the trends for the millennials and surrounding generation to understand the rises of trends in Belgium. The millenials are higly connected through social media and mobile data. The generation Z is even more. As the first members of the generation Z will turn 21 years old in 2017, their influence will impact the market.\\
This super-connection leads to promotion over social media, platformless shopping and digitalisation. Overall, what is needed is a fast process, with instant notification and picture based description of product\cite{stbe}.



\subsection{Environmental}

Belgium has high density of population which impacts its environment particularly in the major cities od the country. Overall, Belgium is oriented toward an environment friendly approach, being ranked 41 out of 180 in the environment protection index\cite{epi} and Belgium has one the most efficient recycling process, in Flanders, 75 \% of the residential waste produced are reused, recycled or composted\cite{wastemana}.

\subsection{Technological}
As \bp is designed to be an on demand application present on desktop computer but mainly mobile device, several technological factor have an impact on the viability of the business.\\

Belgium has a well developed internet infrastructure and rank itself among the most connected country in the world. Belgium ha 8.6 millions of intent users, which represent 82.0\% of the population.\cite{intuser} There 3.6 millions users of fixed broadband and 3.5 millions subscribers of mobile broadband.\cite{intsub} The global coverage of houses is 99.96\% for a 1 Mpbs connection and 91.1\% for a 100 Mbps one.\cite{fixcov} Regarding the mobile coverage, the whole country is covered in 2G connection, almost all the country has access to the 3G network and most urban area have 4G connection.\cite{mobcov}\\

\bp would also rely on cloud technology for the host of the application and the needed computing power. There is lot's of offer of hosting available and it does not rely on the location of the use of the application since Belgium has a fast strong internet coverage.\\

\bp would be part of the on demand economy. This model is based around online platform where independent sellers have an offer for another individual. Classic example of this economy are \textit{Uber} and \textit{eBay}. Recent surveys and data show that this segment is growing and attracting more and more people, not just a young or wealthy population.\cite{odegrow}

\subsection{Conclusion}
This analysis allow us to see that Brussels is a suitable market for \bp.\\
From an economical point view, Brussels' population is wealthy enough to embrace the product. A big part of Brussels' economy comes from the service, which is an industry is often require movement by car. Moreover, a lot of employee in Brussels come from outside the city.\\
From the political side, Belgium's government effectiveness is high and there is close to no corruption. Thus \bp would be safe from any sort of blackmail and its legalisation and regulation process would be on time. The business opportunity does not rely on a lot employment thus the employment law are aligned with it.\\
As the Belgium's population is not and old one and it is still growing. Thus the market size is not threatened. The young generation is increasingly tech-savvy which is correlated to the on-demand economy.\\
Environmentally speaking, Belgium's is very conscious. As \bp will reduce non utilised space, improve parking utilisation and prevent potentially new parking lot to emerge. Thus its aim is linked to environmental issues.\\
Finally, Belgium is technologically able to receive the on-demand business. Indeed, the country is very well connected and hosting service are easily available within.

\section{Industry Description and Market Boundaries}
The parking industry is not a straightforward one, it can be free, private or publicly owned.\\
The obvious business offer is the parking offered through the big building designed only for that goal. Those are usually privately owned and are offer spot only specific location, usually key location with lot's of demand.\\
Then there is the free parking in the street. When a driver wants to find a spot in the street, he would rely on luck and knowledge of the neighbour. In rural or low density area, finding a spot is not a problem as there is low demand. On the other hand, in dense area, the available spots are rather rare because in front of a house there would two spot for ten or more habitant, leading to an offer too small. Free parking are sometime subject to time limit in crowded location, the driver could stay only two hour per example. This process is used to facilitate car turnover.\\
In dense area, the parking are usually subject to charge for non resident of the area. This was developed as a mean to give more liberty to the habitant and incentive people to use another mean of transportation than the car.\cite{nycar} In the same way that for the free parking, the maximum time of spot location is sometime bounded to a few hours to increase the number of spot available.\cite{bxpay} \\
The other possibility is to buy a parking space in the street, this option is usually linked to a house. A similar process is to rent the space on a monthly basis. Those are approaches that are suitable for long term use, typically when the user lives next to the parking spot and struggle to find free ones.

\subsection{Private Parking Short-Term Rent}
Parking problem arise in dense area, where there is a lot of habitant in the neighbour and too few available spot for all of them. A solution to this problem is to charge for parking spots for non resident.\\
First of all, this solution is not perfect, the driver may still struggle to find a spot. Secondly, the parking would be not be free for the user if he does not live in the area.\\

On the other hand, there are lot's of empty private parking spaces. Indeed, houses often come with a parking and a parking space but the owned might not have a car, leading to an empty space. More likely, the owner would not use the spot all the time. Per example, he works in an office, his parking spot would typically be free from 9 am to 5 pm.\\

When connecting those two fact, a solution for the parking shortage arise and lead to a non zero sum game. Indeed, if the parking owner rent his space when he does not use it, he would earn money, and the driver would be able to rent the space thus finding a spot easily and he would have to pay anyway to park his car. It is assumed that the driver would have to pay as he would be in a situation where he struggle to find a sport, thus in a dense area, thus a location where the parking is charged.\\

This is ultimately a private parking short-term rent solution.

\subsection{Market Segment}
\bp has to offer its service in dense area. The application will offer its service only in Brussels at first.\\

The choice of focusing only on one region is based on the fact that the application will need network effects to be successful. Thus heavy initial promotion is needed. The goal is to implement the offer successfully in Brussels first and then explore other accurate location. Offering the private parking short-term rent everywhere, including non dense area, would have bad impact on the image of the application. Indeed, if a user sees that there is no availability in his neighbourhood, he is unlikely to use it again, although there was no availability in his neighbourhood because there was no need.\footnote{Assumption, to be verified with a survey} \\

Choosing Brussels as a the first city to implement to project is an appropriate choice. Indeed, Brussels is a leading city in Europe. The city is dense. There is 700 000 cars in movement at peak hour for 265 000 available spots.\cite{parkbx} The macro environment presented how Brussels is suitable for \bp. The following map shows the parking rules in Brussels\cite{parkbx} : \\

\begin{figure}[h]
\centering
\caption{Map of Brussels' regulated parking area}
\label{bxmap}
\includegraphics[keepaspectratio=true,width=\textwidth-2cm]{images/bxpark.png}
\end{figure}


The blue area are free but for a length of 2 hours maximum and in the green, grey, orange and red ones, the driver need to pay to park his car. The \autoref{bxfare} details the price and the maximum time in each zone.\\

\begin{table}[h]
\centering
\caption{Brussels' parking fare}
\label{bxfare}
\begin{tabular}{|l|l|l|l|l|}
\hline
           & Green & Grey & Orange & Red  \\ \hline
Max        & /     & 4:30 & 2:00   & 2:00 \\ \hline
0:30       & 0.50  & 0.50 & 0.50   & 0.50 \\ \hline
1:00       & 1.00  & 1.00 & 1.00   & 2.00 \\ \hline
1:30       & /     & /    & 2.00   & 3.50 \\ \hline
2:00       & 3.00  & 3.00 & 3.00   & 5.00 \\ \hline
3:00       & 4.50  & 5.00 & /      & /    \\ \hline
4:00       & 6.00  & 8.00 & /      & /    \\ \hline
4:30       & /     & 9.50 & /      & /    \\ \hline
Extra hour & 1.50  & /    & /      & /    \\ \hline
\end{tabular}
\end{table}

As a city with an heavily regulated parking policy, Brussels is the ideal candidate for \bp to start its growth.

\subsection{Customer}
The service is based on two types customer.\\

In the first place, there is the parking space renter. In order to be able to rent your parking spot, one would need to own a parking spot in Brussels. As the parking spot should be free at a recurring schedule in order to rent it consistently, employed people who do not work at home and use the car to go their work place is an accurate person. One could assume that renting its spot would only attract not wealthy people but renting its spot is also a ecologist act.\footnote{survey} There is also benefit of renting a space if you are a user of the application on the other side as well. Indeed, there would be bonus for space renters.\\

The second type of customer is the one willing to rent a place, the tenant. The people who might fall into that category need to use a car, whether it is owned or leased, in Brussels. They also need to have a smart phone and a mobile internet connection.\\
Whether this group of people will use the service or not is related to any kind of wealth factor. Indeed, if a driver is looking for a spot and none are available for the period he desire, he cannot pay extra for a free spot, there are none available. If someone on the poor segment of the population is looking for a spot, he will want to find one and pay if needed as he is already in the location and going back home without completing the purpose of the ride is unlikely to be a better option.\\
More than just owning a smart phone, the typical user has to be aware that such technology exists. A person who know and used at least once \textit{Uber} would fall into such a category.

\subsection{Suppliers of the Industry}
The supplier of the parking industry as whole are parking sport. For \bp it is precisely private parking spots.\\
The particularity of the business model is that the supplier is also a customer. Indeed, it will be the owner of the parking spot that will have to register itself in the application an enter its parking spot in the system.\footnote{Should I talk about hosting service as a supplier ?}

\subsection{Competitors}
\subsubsection{Parking Offer Comparison}
Before analysing \bp 's competitor in its particular business model, it seems appropriate to compare the different parking offer and why \bp 's offer is relevant.\\

The \autoref{parkof} propose a comparison of the different means available to a motorist to park his car. Commercial parking design big parking lot owned by a private company, private long term parking is parking owned or rented on a monthly basis, finally private parking short term represent \bp 's offer.

\begin{table}[h]
\centering
\caption{Parking offer comparison}
\label{parkof}
\begin{tabular}{l|l|l|l|l|l|}
\cline{2-6}
\multirow{2}{*}{}                           & \multicolumn{2}{c|}{\textbf{Public Parking}}                                                      & \multicolumn{1}{c|}{\multirow{2}{*}{\textbf{\begin{tabular}[c]{@{}c@{}}Commercial\\ Parking\end{tabular}}}} & \multicolumn{1}{c|}{\multirow{2}{*}{\textbf{\begin{tabular}[c]{@{}c@{}}Private Parking\\ Long Term\end{tabular}}}} & \multicolumn{1}{c|}{\multirow{2}{*}{\textbf{\begin{tabular}[c]{@{}c@{}}Private Parking\\ Short Term\end{tabular}}}} \\ \cline{2-3}
                                            & \multicolumn{1}{c|}{\textbf{Free}} & \multicolumn{1}{c|}{\textbf{Regulated}}                      & \multicolumn{1}{c|}{}                                                                                       & \multicolumn{1}{c|}{}                                                                                              & \multicolumn{1}{c|}{}                                                                                               \\ \hline
\multicolumn{1}{|l|}{\textbf{Stay}}         & Very Short                         & Very Short                                                   & \begin{tabular}[c]{@{}l@{}}Short and\\ Long\end{tabular}                                                    & Long                                                                                                               & Short                                                                                                               \\ \hline
\multicolumn{1}{|l|}{\textbf{Price}}        & Free                               & \begin{tabular}[c]{@{}l@{}}Cheap to\\ Expensive\end{tabular} & Expensive                                                                                                   & Moderate                                                                                                           & Cheap                                                                                                               \\ \hline
\multicolumn{1}{|l|}{\textbf{Location}}     & Scarce                             & Frequent                                                     & Scarce                                                                                                      & Scarce                                                                                                             & Frequent                                                                                                            \\ \hline
\multicolumn{1}{|l|}{\textbf{Availability}} & High to Low                        & High to Low                                                  & High                                                                                                        & Low                                                                                                                & Moderate                                                                                                            \\ \hline
\multicolumn{1}{|l|}{\textbf{Speed}}        & Fast                               & Fast                                                         & Slow                                                                                                        & Fast                                                                                                               & Fast                                                                                                                \\ \hline
\end{tabular}
\end{table}

From this comparison, it appears clearly that the each mean has a different purpose. Public parking is clearly aimed at very short term stay. Public parking is available all the time as the spots might all be used. Thus, if someone need a short term stay, he would use a commercial parking but this offer is scarce, usually parking lot a present only several key location, moreover they are costly.\footnote{check price} The short term private parking thus make sense as it should be available (once the product is settled) and cheaper than the original parking. To summarise, this option would offer very short term stay cheaper or when there is no spot available and short term stay when it is not possible through public parking (e.g blue zone, free but a stay of maximum 2 hours). Another advantage is that it is fast compared to a commercial parking. In the commercial parking there all the ticket procedure and then you have to leave the parking which can take time.\\

The long term private parking has another purpose, to park the car for a long time or every day at the same spot. This parking option is not captured by \bp although having a private parking is not always possible and \bp could thus help the driver to find a place every day.

\subsubsection{Short Term Private Parking Offers}

\begin{itemize}
\item ShareMyPark
\item JustPark
\item other only for a day
\item parkopedia, commercial parking
\item kerb not present in brussels
\item mobypark both type of parking, few location
\end{itemize}

\subsection{Porter’s Five Forces}

\subsection{Internal Analysis}

\subsection{SWOT}

\chapter{Company and Product Description}

\chapter{Marketing Plan}

\chapter{Operation and Team}

\chapter{Critical Risks}

\chapter{Financial Plan}


\bibliographystyle{unsrt}
%\bibliographystyle{plain}
\bibliography{./reference/bibliography}


\end{document}
